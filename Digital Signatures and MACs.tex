\section{Digital Signatures and Message Authentication Codes}
\\Goal: Send a message with an authentication code to prevent tampering
\begin{itemize}
\item Signature Function
    \begin{itemize}
    \item $\sigma = s =$ sign(m)
    \item Public Key Encryption
    \\ Private key: signing key, Public key: verification key
        \begin{itemize}
        \item $\sigma = \text{sign}_{priv}(m)$
        \item $\text{bool verify}_{pub}(m, \sigma)$
        \end{itemize}
    \item Breaking a signature scheme; adversarial model
        \begin{itemize}
        \item Key-only attack: access to public key
        \item Known message attack: access to messages and their signatures (no choice)
        \item Chosen message attack: Choose specific message signature pairs. Strictly more powerful than those above. Oracle. All requests in one shot.
        \item Adaptive chosen message attack: Change messages to sign after each response from the oracle.
        \end{itemize}
    \item Attack models
        \begin{itemize}
        \item Total Break: found private key
        \item Universal Forgery: Algorithm is efficient, equivalent to private key
        \item Selective Forgery: Alice chooses messages that can be forged
        \item Existential Forgery: Able to forge a signature for some message
        \end{itemize}
    \item Existential Forgery Example with RSA
        \begin{itemize}
        \item Signer: (Private) $p, q, d = e^{-1} \text{ mod } \phi$
        \item Verifier: $e,n$
        \item $\sigma = m^d \text{ mod } n$
        \\verify = $\sigma^e =?\,m$
        \item Forgery
            \begin{itemize}
            \item $\sigma = \text{Random String}$; $\sigma \in_R Z_N^*$
            \item $m = \sigma^e\,\text{mod}\,n$
            \item (message, signed message) = $(\sigma^e, \sigma)$
            \end{itemize}
        \end{itemize}
    \item Existential Forgery Fix
        \begin{itemize}
        \item $\sigma = h(m)^d \text{ mod } n$
            \begin{itemize}
            \item Where h is collision, PIR, 2nd PIR resistant. 
            \end{itemize}
        \item Verify: $\sigma^e = h(m) \text{ mod } n$
        \end{itemize}
    \end{itemize}
\item Message Authentication Codes (MAC)  
\begin{itemize}  
    \item Parameters  
    \begin{itemize}  
        \item One key k; $\sigma = \text{sign}_k(m)$; $\text{bool verify}_k(m,\sigma)$ 
    \end{itemize}  
    \item $\text{MAC}_k(m) = h(k||m)$ is BAD  
        \begin{itemize}  
        \item Forgeable  
        \item Hash Functions 
            \begin{itemize} 
            \item Pre-Process: $y = \text{pre-process}(m)$; add padding         \item Compression: $y = \text{compress}(y, f(m))$; continue to consume $f(m)$  
            \item Post-Process: output $g(y)$  
            \end{itemize}  
        \item Given a known message attack: $(m, \sigma(m))$
            \begin{itemize}
            \item h(x) = 
                \begin{itemize}
                \item pre-process(); compress(x); post-process();
                \end{itemize}
            \item $\text{MAC}_k(m||m') = \text{compress}(MAC_k(m),m');$
                \\$\text{                post-process()}$;
            \end{itemize}
    \end{itemize}  
    \item Encryption != Signature Authentication
        \\ex. $\text{CBC-MAC}_K(m)$ outputs the last block
    \item Real World Example: WSJ password file  
        \begin{itemize}
        \item $\text{MAC}_k(\text{username}) = \text{crypt}(\text{user }||k)$
        \item Unfortunately, crypt truncated the user input after 8 characters
        \item Thus, inputting a 7 char username made it trivially easy to generate the first character of the secret key. (and then you could generate more and more elements...)
        \end{itemize}
    \end{itemize}
\end{itemize}