\section{Block Ciphers}
\begin{itemize}
\item Kirchoff's Law
    \begin{itemize}
    \item Security of algorithm is based on the secret key rather than the algorithm.
    \item Assume nothing about the secrecy of the algorithm
    \end{itemize}
\item Block Cipher Basics
    \begin{itemize}
    \item Block cipher: Plaintext $\rightarrow$ Ciphertext on a fixed N-bit size. 
    \item Symmetric Key cryptography; treat algorithm as a block box
    \item k-bit key, n-bit message, c-bit ciphertext
    \item Brute force key: $O(2^k)$
    \end{itemize}
\item Building a table
    \begin{itemize}
    \item $\sum^K x \sum^k \rightarrow \sum^n$
    \item Permutations from $\sum^n \rightarrow \sum^n$?
        \begin{itemize}
        \item $2^n!$
        \item Sterling's Approximation: $x! = \sqrt{2\pi x}\,(\frac{x}{e})^x $
        \end{itemize}
    \item How big does a table need to be to map $\sum^n \rightarrow \sum^n$? $log_2(x!)$ where $x = 2^n$.
        \begin{itemize}
        \item If n = 128 bits; $k = log_2(2^{128}!) = 2^{134} bits.$
        \end{itemize}
    \end{itemize}
\item Real-World Block Ciphers
\begin{itemize}
    \item Data Encryption Standard (DES)
        \begin{itemize}
        \item NIST Standard
            \begin{itemize}
            \item 56-bit keys (due to 8th bit of every byte being a parity bit)
            \item 64-bit block size
            \end{itemize}
        \item Invented by IBM
        \item Attack: Linear Cryptanalysis - Relationships between plaintext & ciphertext (supposedly NSA improved DES to be resistent to this attack)
        \item Replaced by AES
        \end{itemize}
    \item Advanced Encryption Standard (AES)
        \begin{itemize}
        \item Rijndael
            \begin{itemize}
            \item 128-bit blocksize
            \item 128/192/256-bit keysizes
            \end{itemize}
        \item Fast + low gate count
        \end{itemize}
    \item RC5: Ron's Code 5
        \begin{itemize}
        \item Simple, fast
        \item 32/64/128-bit block size
        \item 0-2040-bit key size
        \end{itemize}
    \end{itemize}
\item Strength of a block Cipher
    \begin{itemize}
    \item Data Complexity: Known-plaintext; how many known-plaintexts does the adversary need to get?
    \item Storage Complexity: How much hard drive space does the adversay need?
    \item Computational Complexity: How much time does it take the adversary to process the key?
    \end{itemize}
\item Breaking Block Ciphers
    \begin{itemize}
    \item Brute Force Guess
        \begin{itemize}
        \item What if k is small?
        \item $O(2^k)$ where k is the bits of the key
        \item Known-plaintext attack: Get plaintext/CT pair
        \end{itemize}
    \item Small block size attack
        \begin{itemize}
        \item What if n is small?
        \item Known PT attack: Create a table of PT $\rightarrow$ CT w/o key
        \item Birthday Paradox: Proportional to the square root of the size of the search space
            \begin{itemize}
            \item $\sqrt(2^n) = 2^{n/2}$ known plaintexts/ciphertext pairs
            \item Put these plaintext/ciphertext pairs in a table
            \item Then you listen to another $2^{n/2}$ ciphertexts.
            \item High probability that one of the rows will hit (Key-equivalent translation).
            \end{itemize}
        \end{itemize}
    \end{itemize}
\item Double Encryption: $E_{k2}(E_{k1}(m))) = c$
    \begin{itemize}
    \item This is not $O(2^{2k})$ complexity.
    \item Meet in the Middle Attack: $O(2^k)$ space $\rightarrow O(2^k)$ time.
    \item Method
        \begin{itemize}
        \item Encrypt plaintext with each possible $k_1$ ($2^k$ keys).
        \end{itemize}
    \end{itemize}
\end{itemize}